\documentclass[12pt, a4paper]{article}
\usepackage{comment}
\usepackage{lmodern}
\usepackage[inline]{enumitem}
\usepackage{xcolor}
\usepackage{blindtext}
\usepackage{scrextend}
\usepackage{cmap}
\usepackage[czech]{babel}
\usepackage[T1]{fontenc}
\usepackage[utf8]{inputenc}
\usepackage{graphicx}
\usepackage[capposition=bottom]{floatrow}
\usepackage{float}
\usepackage{amsmath}
\usepackage{pdfpages}
\usepackage{hyperref}
\begin{document}

% Pouze informace
\graphicspath{ {img/} }

% Úvodní stránka
\thispagestyle{empty}
\begin{center}
\begin{minipage}{0.75\linewidth}
    \centering
%University logo
    \vspace{3cm}
    \includegraphics[width=0.75\linewidth]{fav-logo.pdf}\\
    \vspace{0.5cm}
%Thesis title
    {\uppercase{\Large KIV/UPS \\ \textbf{Online hra - Pong!}\par}}
    \vspace{3cm}
%Author's name
    {\Large Jakub Vítek (viteja) - A19B0222P\par}
    \vspace{2cm}
%Degree
    \vspace{1cm}
%Date
    {\Large Leden 2020}
\end{minipage}
\end{center}
\clearpage
\newpage

% Část obsahu dokumentu
\tableofcontents
\newpage

\section{Hra}
\paragraph{}
Původní Pong byla jedna z nejstarších her. Jednalo se o tenisovou 2D hru, která byla původně vydána společností
Atari v roce 1972.

\paragraph{}
Hratelnost této hry je velice jednoduchá. Dva hráči na opačných stranách ovládají plošiny, kterými odráží míček. Ve
chvíli kdy jednomu z hráčů propadne míček, druhý hráč získává bod. Míček se odraží od stěn a od plošin hráčů, v
případě odražení od hráčské plošiny je míček navíc mírně zrychlen.


\newpage
\section{Implementace}
\subsection{Server}
\paragraph{}
Serverová část byla implementována v jazyku go od společnosti Google. Důvodem pro výběr tohoto jazyka bylo dynamické
typování, automatická správa paměti, pohodlnější práce s řetězcem než v jazyce C a jednoduchost paralelizace.

\subsubsection{Komunikace}
\paragraph{}
Komunikace serveru je řešena samostatným vláknem, které řeší základní tvorbu socketů a čtení dat z nich. Server
využívá hlavního socketu pro připojení nových klientů. Při připojení nového klienta se vždy vytvoří nový socket, přes
který je s klientem komunikováno.

\paragraph{}
Komunikace s vícero sockety v jednom vlákně je řešena systémovým voláním operačního systému GNU/Linux \textit{select},
ktere naplní předpřipravenou sadu socketů informací o tom, zda některý ze socketů komunikuje. Následně jsou všechny
komunikující sockety postupně obslouženy. Při vytvoření nového spojení (socketu) se na serveru vytvoří a uloží do mapy
struktura \textit{Client}, tato struktura také obsahuje buffer přijatých dat. Obsluhou klientského socketu je pak
myšleno přečtení příchozích dat a jejich uložení do toho bufferu.

\paragraph{}
Pro každý vytvořený klient je spuštěn dekodér (jehož funkce je popsána v sekci protokol) jehož prací je čtení dat
uložených v bufferu a jejich překlad na zprávy a jejich uložení do bufferu pro zprávy.

\subsubsection{Zprávy a akce}
\paragraph{}
Při startu server se vytváří \textit{Manager}, jehož běh probíhá v samostatném vlákně. Při inicializaci této
struktury se registrují akce do mapy akcí, kde klíčem je typ zprávy a hodnotou odkaz na funkci, která má být spuštěna
při přijetí tohoto typu zprávy. Typ zprávy je možné přečíst z hlavičky zprávy. Ve chvíli, kdy zpráva na serveru
existuje, je již ověřeno, že má validní hlavičku a formát, neověřuje se však její obsah.

\paragraph{}
Na serveru jsou zprávy dvojího typu, zprávy herní a kontrolním. Kontrolní zprávy obvykle řeší akce jako například
registraci či jinou správu uživatele či hry (třeba její založení či opuštění). U herních zpráv je ověřeno, zda je
uživatel registrován a splňuje všechny požadavky na přesun zprávy do bufferu herního serveru. V případě, že zpráva
požadavky nesplňuje, je vrácena uživateli chybová správa se stejným typem.

\subsubsection{Herní server}
\paragraph{}
Herní server je běžící smyčkou ve samostatném vlákně. Běžící server může být ve stavu \textit{PAUSED} či
\textit{UNPAUSED}. Uživatel není schopný pauzu sám o sobě vyvolat. Server je ve stavu pauzy v případě, že je ve hře
připojen jeden hráč či jeden z hráčů ztratil spojení.

\paragraph{}
Smyčka herního serveru proběhne třicetkrát za sekundu. Nejprve se ověřuje, zda jsou všichni uživatelé připojení a
nemají ztracené spojení - ztráta spojení je detekována jako prodleva větší jak 2 vteřiny od poslední zprávy. Následně
jsou přijaty všechny uživatelské zprávy změny hráčské pozice. U těchto zpráv je provedena korekce a ověřování před
zapsáním nové pozice.

\paragraph{}
V dalším kroce je proveden tick míče, aktualizace jeho pohybu, detekce kolizí. Následně je vytvořena zpráva s
aktuálním stavem hry, která je odeslána připojeným hráčům. V poslední řade je ověřován konec hry. V případě, že
nějaký z hráčů dosáhne 10 bodů, je herní smyčka ukončena a hráči jsou informování o výhře. Po ukončení hry je hra
smazána a vlákno ukončeno.

\newpage
\subsection{Klient}
\paragraph{}
Implementace herního klienta je provedena v jazyku Python (verze minimálně 3.7). Klient je realizován za pomocí
multiplatformní sady \textit{pygame}. Všechna herní menu byla vytvořena za použití knihovny PyGameMenu. Alert boxy
byly vytvořeny za využití modulu tkinter.

\paragraph{}
Při spuštění herního klienta je uživatel přivítán menu pro připojení. Zde je možné připojit se k běžícímu serveru po
vyplnění IPv4 adresy, portu, herní přezdívky. Připojení vytváří nového uživatele. Znovupřipojení požádá server o
přihlášení k již existujícímu uživatelskému účtu. Při registraci či připojení k serveru klient aktivně čeká maximálně
dvě sekundy na odpověd od serveru, pokud mu v této době odpověd nepříjde nebo přijde záporná, odpojí se od serveru
(pokud bylo spojení navázáno)

\paragraph{}
Po přihlášení se uživatel dostane do menu s lobby, kde může tlačítky vytvořit novou hru, obnovit seznam existujících
her či se připojit k nějaké hře k seznamu. Po připojení uživatel začne posílat KeepAlive zprávy, které server
informují od tom, že je spojení v pořádku. Odpověd na tyto zprávy také ujišťuje uživatele, že je stále připojen k
serveru. Poslední možností uživatele v tomto menu je informování serveru o tom, že se klient bude odpojovat (validní
ukončení spojení). Při ukončení spojení nebo jeho ztrátě je uživatel vrácen zpět do menu sloužícího k připojení.

\paragraph{}
V případě že si hráč vytvoří do hry nebo se připojí do hry již existující, je hráči přidělena herní pozice a spuštěna
herní smyčka. Hra je zobrazována tvz. "na výšku". Hráč který se připojí jako první je hráč horní, druhá hráč je
umístěn do spodní části herní plochy. Oba hráči se mohou pohybovat pouze doleva a doprava tak, aby nemohli opustit
herní pole. Hratelný hráč je zobrazen jako čára zeleně, protivní je zobrazen v barvě červené. Míč je zobrazen jako
fialový vyplněný kruh. Hra také vykresluje bíle středovou čáru, okolo které jsou na pravé straně vykresleny body
jednotlivých hráčů. V případě, že je hra pozastavena, je ve středu obrazovky vypsán text PAUSED.

\subsubsection{Herní smyčka a komunikace}
\paragraph{}
Jakýkoliv požadevek v menu aktivně čeká na odpověd od serveru nebo vypršení času, teprve poté je možné odeslat
požadavek nový. Herní klient rozeznává v zásadě tři typy zpráv. Jedním typem je odpověd serveru na požadavek
KeepAlive, který resetuje čas poslední komunikace. Druhým typem jsou zprávy řízení klienta, v rámci těchto zpráv se
řeší připojení k serveru, registrace, odpověd na založení hry či připojení ke hře jiné. Poslední zprávy jsou zprávy
herní, které mají svůj vlastní buffer.

\paragraph{}
V herní smyčce se nejprve ověřuje zda je stále platné připojení k serveru. Následně se procházejí přijaté herní
zprávy a odpovědi od serveru na základě, kterých jsou spouštěny akce. Následně ověřujeme, zda se uživatel nechce
pohnout, v případě že ano, pohyb ověříme a odešleme na server. V poslední řade aktualizujeme data a vykreslíme je
(pozice hráčů, míčku, počet bodů, atd.).

\paragraph{}
Stejně jako na serveru je klient omezen na maximálně 30 průchodů smyčky za sekundu, díky čemuž klientu může dosahovat
maximálně 30FPS.

\newpage
\section{Protokol}
\paragraph{}
Protokol vytvořený pro komunikace klientů se serverem je protokolem textovým (bytovým). Prokol vyhrazuje některé
znaky (byte) jako znaky kontrolní.

    \begin{table}[H]
        \centering
        \begin{tabular}{|c|l|}
            \hline
            Znak & Popis \\
            \hline
            \hline
            < & Začátek zprávy \\
            \hline
            > & Ukončení zprávy \\
            \hline
            | & Oddělovač hlavičky a obsahu \\
            \hline
            : & Oddělovat klíče a hodnoty páru \\
            \hline
            ; & Oddělovač párů \\
            \hline
            \textbackslash & Escape znak \\
            \hline
        \end{tabular}
        \caption{Seznam řídících znaků}
    \end{table}

\paragraph{}
Zpráva se skládá z hlavičky a obsahu. Hlavička zprávy je vždy pevně daná a musí, v tomto pořadí, obsahovat
\textit{identifikátor zprávy}, \textit{návratový identifikátor}, \textit{typ zprávy} a dekodér na serveru si ke
zprávě sám přidává identifikátor odesílatele. Hlavičku nelze escapovat. Hodnota \textit{"id"} určuje pořadí zprávy.
Hodnota \textit{"rid"} je nenulová pouze v případě, kdy je předpoklad, že se na zprávu bude odpovídat. Pokud klient
pošle zprávu s RID 10, server odpoví se zprávou ID 10.

\paragraph{}
Obsah zprávy je tvořen dvojicí klíč a hodnota, které jsou odděleny znakem \textit{":"}. Dvojice je ukončena znakem
\textit{";"}. U obsahu nezáleží na pořadí klíčů, důležitá je pouze jejich existence. Obsah lze escapovat znakem
\textit{"\textbackslash"}, pro napsání escapovacího znaku je třeba napsat escapovací znak dvakrát za sebou (tj.
\textit{"\textbackslash\textbackslash"}).

\paragraph{}
Zpráva vždy začíná počátečním znakem \textit{"<"} následovaným dvojicí ve formátu \textit{"id:\%d"}. Tu následující
další dvě dvojice \textit{"rid:\%d"} a \textit{"type:\%d"}. Jednotlivé dvojice jsou oddělené znakem \textit{";"}.
Následně je zde umístěn znak konce hlavičky \textit{"|"}. Dvojice v obsahu zprávy jsou pak načítány při splnění
formátu \textit{"key:value;"} do nalezení koncového znaku \textit{">"} nebo přetečení maximální velikosti zprávy.


\newpage
\subsection{Seznam zpráv - klient}
\subsubsection{KeepAlive}
\textbf{Typ zprávy: } 1100 \newline
\textbf{Formát: } \newline  <id:INT;rid:INT;type:1100;|status:ok;> \newline
Klient periodicky odesílá zprávu na server. Server posílá stejnou zprávu zpět. \newline

    \begin{table}[H]
        \centering
        \begin{tabular}{|c|c|c|}
            \hline
            Klíč & Datový typ & Hodnota \\
            \hline
            \hline
            status & string & ok \\
            \hline
        \end{tabular}
        \caption{Obsah KeepAlive zprávy - klient i server}
    \end{table}

\subsubsection{GameJoin}
\textbf{Typ zprávy: } 2100 \newline
\textbf{Formát: } \newline  <id:INT;rid:INT;type:2100;|gameID:INT;playerID:INT;> \newline
Klient odesílá požadavek o připojení do hry. \newline

    \begin{table}[H]
        \centering
        \begin{tabular}{|c|c|c|}
            \hline
            Klíč & Datový typ & Hodnota \\
            \hline
            \hline
            gameID & int & \%d \\
            \hline
            playerID & int & \%d \\
            \hline
        \end{tabular}
        \caption{Obsah GameJoin zprávy - klient}
    \end{table}

    \begin{table}[H]
        \centering
        \begin{tabular}{|c|c|c|}
            \hline
            Klíč & Datový typ & Hodnota \\
            \hline
            \hline
            status & string & ok || error \\
            \hline
            msg & string & \%s \\
            \hline
            player & string & "1" || "2" \\
            \hline
        \end{tabular}
        \caption{Obsah GameJoin zprávy - server}
    \end{table}

\subsubsection{GameCreate}
\textbf{Typ zprávy: } 2000 \newline
\textbf{Formát: } \newline  <id:INT;rid:INT;type:2000;|playerID:INT;> \newline
Klient odesílá požadavek o vytvoření hry. \newline

    \begin{table}[H]
        \centering
        \begin{tabular}{|c|c|c|}
            \hline
            Klíč & Datový typ & Hodnota \\
            \hline
            \hline
            playerID & int & \%d \\
            \hline
        \end{tabular}
        \caption{Obsah GameCreate zprávy - klient}
    \end{table}

    \begin{table}[H]
        \centering
        \begin{tabular}{|c|c|c|}
            \hline
            Klíč & Datový typ & Hodnota \\
            \hline
            \hline
            status & string & ok || error \\
            \hline
            msg & string & \%s \\
            \hline
            gameID & int & \%d \\
            \hline
        \end{tabular}
        \caption{Obsah GameCreate zprávy - server}
    \end{table}

\subsubsection{GameList}
\textbf{Typ zprávy: } 2300 \newline
\textbf{Formát: } \newline  <id:INT;rid:INT;type:2300;|playerID:INT;> \newline
Klient odesílá požadavek o seznam existujících her. \newline

    \begin{table}[H]
        \centering
        \begin{tabular}{|c|c|c|}
            \hline
            Klíč & Datový typ & Hodnota \\
            \hline
            \hline
            playerID & int & \%d \\
            \hline
        \end{tabular}
        \caption{Obsah GameList zprávy - klient}
    \end{table}

    \begin{table}[H]
        \centering
        \begin{tabular}{|c|c|c|}
            \hline
            Klíč & Datový typ & Hodnota \\
            \hline
            \hline
            status & string & ok || error \\
            \hline
            msg & string & \%s \\
            \hline
            gameCount & int & \%d \\
            \hline
            gameID\%d & int & \%d (více klíčů) \\
            \hline
        \end{tabular}
        \caption{Obsah GameList zprávy - server}
    \end{table}

\subsubsection{Reconnect}
\textbf{Typ zprávy: } 2200 \newline
\textbf{Formát: } \newline  <id:INT;rid:INT;type:2200;|username:STRING;> \newline
Klient odesílá požadavek o znovupřipojení na existující účet. \newline

    \begin{table}[H]
        \centering
        \begin{tabular}{|c|c|c|}
            \hline
            Klíč & Datový typ & Hodnota \\
            \hline
            \hline
            username & string & \%s \\
            \hline
        \end{tabular}
        \caption{Obsah Reconnect zprávy - klient}
    \end{table}

    \begin{table}[H]
        \centering
        \begin{tabular}{|c|c|c|}
            \hline
            Klíč & Datový typ & Hodnota \\
            \hline
            \hline
            status & string & ok || error \\
            \hline
            msg & string & \%s \\
            \hline
            gameID & int & \%d \\
            \hline
            playerID & int & \%d \\
            \hline
            playas & string & "1" || "2 \\
            \hline
        \end{tabular}
        \caption{Obsah Reconnect zprávy - server}
    \end{table}

\subsubsection{Register}
\textbf{Typ zprávy: } 1000 \newline
\textbf{Formát: } \newline  <id:INT;rid:INT;type:1000;|name:STRING;> \newline
Klient odesílá požadavek o registraci. \newline

    \begin{table}[H]
        \centering
        \begin{tabular}{|c|c|c|}
            \hline
            Klíč & Datový typ & Hodnota \\
            \hline
            \hline
            name & string & \%s \\
            \hline
        \end{tabular}
        \caption{Obsah Register zprávy - klient}
    \end{table}

    \begin{table}[H]
        \centering
        \begin{tabular}{|c|c|c|}
            \hline
            Klíč & Datový typ & Hodnota \\
            \hline
            \hline
            status & string & ok || error \\
            \hline
            msg & string & \%s \\
            \hline
            playerID & int & \%d \\
            \hline
        \end{tabular}
        \caption{Obsah Register zprávy - server}
    \end{table}

\subsubsection{GameAbandod}
\textbf{Typ zprávy: } 2500 \newline
\textbf{Formát: } \newline  <id:INT;rid:INT;type:2500;|playerID:INT;> \newline
Klient odesílá informaci o definitivním opuštění hry. Klient ignoruje odpověd. \newline

    \begin{table}[H]
        \centering
        \begin{tabular}{|c|c|c|}
            \hline
            Klíč & Datový typ & Hodnota \\
            \hline
            \hline
            playerID & int & \%d \\
            \hline
        \end{tabular}
        \caption{Obsah GameAbandon zprávy - klient}
    \end{table}

\subsubsection{PlayerUpdateState}
\textbf{Typ zprávy: } 3000 \newline
\textbf{Formát: } \newline  <id:INT;rid:INT;type:3000;|playerID:INT;x:INT;y:INT;paused:STR-BOOL;> \newline
Klient odesílá informace pohybu na server. Server odpovídá zprávou 2400. Deprecated hodnoty jsou ignorovány serverem
ale stále vyžadovány protokolem. \newline

    \begin{table}[H]
        \centering
        \begin{tabular}{|c|c|c|}
            \hline
            Klíč & Datový typ & Hodnota \\
            \hline
            \hline
            playerID & int & \%d \\
            \hline
            x & int & \%d \\
            \hline
            y & int & \%d (deprecated) \\
            \hline
            paused & string & "true" || "false" (deprecated) \\
            \hline
        \end{tabular}
        \caption{Obsah PlayerUpdateState zprávy - klient}
    \end{table}

\newpage
\subsection{Seznam zpráv - server}

\subsubsection{GameUpdateStateClient}
\textbf{Typ zprávy: } 2400 \newline
Server odesílá informace o aktuálním stavu hry (samovolně ale hlavně jako reakci na zprávu 3000). \newline

    \begin{table}[H]
        \centering
        \begin{tabular}{|c|c|c|}
            \hline
            Klíč & Datový typ & Hodnota \\
            \hline
            \hline
            player1x & int & \%d \\
            \hline
            player1y & int & \%d \\
            \hline
            player2x & int & \%d \\
            \hline
            player2y & int & \%d \\
            \hline
            score1 & int & \%d \\
            \hline
            score2 & int & \%d \\
            \hline
            ballx & int & \%d \\
            \hline
            bally & int & \%d \\
            \hline
            ballspeed & int & \%d \\
            \hline
            ballrotation & int & \%d \\
            \hline
            paused & string & "true" || "false" \\
            \hline
        \end{tabular}
        \caption{Obsah GameUpdateStateClient zprávy - odesílatel server}
    \end{table}

\subsubsection{GameEnd}
\textbf{Typ zprávy: } 2400 \newline
Server odesílá informaci o konci hry. Ignoruje protesty klienta. \newline

    \begin{table}[H]
        \centering
        \begin{tabular}{|c|c|c|}
            \hline
            Klíč & Datový typ & Hodnota \\
            \hline
            \hline
            status & string & "ok" \\
            \hline
            msg & string & "Player \%d won!" \\
            \hline
        \end{tabular}
        \caption{Obsah GameEnd zprávy - odesílatel server}
    \end{table}

\newpage
\subsection{Kontext klienta}
\subsubsection{Kontext klienta na serveru}
\paragraph{}
Klient po připojení k serveru může v zásadě existovat v několika kontextech. Kontextem je myšleno to, jak server vidí
připojeného klienta. Na základě tohoto kontextu zpracuje různé typy zpráv, které v opačném případě odmítá.
\paragraph{}
Po navázání spojení je klient ve stavu UNREGISTERED. V tomto kontextu server od klienta přijímá pouze zprávy 1000
(REGISTER) a 2200 (RECONNECT). Na všechny ostatní zprávy odpovídá odpovědí obsahující indikaci chybového stavu
\paragraph{}
V případě úspěšného zpracování zprávy ve stavu UNREGISERED se klient pro server stává REGISTERED. Ve stavu
registered server zpracovává zprávy 1100 (KeepAlive), 2000 (GameCreate),
2100 (GameJoin), 2300 (GameList),
\paragraph{}
Po úspěšném založení hry, po připojení k existující hře či znovupřipojení po ztraceném spojení je klient pro server
ve stavu PLAYING. V tomto stavu server zpracovává zprávy 2500 (GameAbandon) a 3000 (UpdatePlayerState).

\subsubsection{Kontext v herním klientu}
Klient implementuje podobný kontext jako na serveru. Klient může být ve stavu NOT-CONNECTED, kdy je uživateli v GUI
zobrazeno menu pro zadání adresy, portu serveru a jména. Po úspěšné registraci je uživatel přesunut do stavu
CONNECTED, kdy je mu zobrazeno menu po připojení do existující hry či založení nové (případně odpojení od serveru).
Posledním stavem je stav PLAYING, kdy se uživateli vykresluje hra synchronizovaná se stavem hry na serveru a uživatel
může odesílat změnu pozice své platformy na server.


\newpage
\section{Spuštění serveru a klienta}
\paragraph{}
Sestavení serveru vyžaduje nainstalované prostředí jazyka Golang v minimální verzi 1.13. Server lze sestavit příkazem
\textit{go build .}, případně rovnou spustit příkazem \textit{go run .} (za předpokladu, že je terminál přepnut do
složky se zdrojovým kódem serveru).

\paragraph{}
Sestavení klienta vyžaduje nainstalovaný interpret jazyka Python v minimální verzi 3.7. Dále je potřeba nainstalovat
moduly pygame (\textit{pip3 install pygame}), PyGameMenu (\textit{pip3 install pygame-menu} a modul tkinter. Klient
pak lze spustit z příkazové řádky příkazem \textit{python3 ./main.py}  (za předpokladu, že je terminál přepnut do
složky se zdrojovým kódem klienta).

\newpage
\section{Závěr}
\paragraph{}
Před vytvářením této semestrální práce jsem měl nulové zkušenosti s vyvíjením her i s vyvíjením síťových aplikací.
Také jsem neměl prakticky žádné zkušenosti s jazykem go. Jedinou výhodou byla moje základní znalost jazyka Python.

\paragraph{}
Vrámci tvorby aplikace jsem několikrát změnil technologii serveru i klienta, což mě stálo spoustu času, který mi na
závěrečné odevzdání práce chyběl. Snažil jsem se proto vytvořit konečné řešení co nejrychleji jsem mohl. Musím však
říci, že s prací nejsem spokojen, po této zkušenosti bych server i client navrhl jiným způsobem.

\paragraph{}
Přesto si však myslím, že jsem práci splnil, tedy aspoň v to doufám...

\paragraph{}

\newpage
\listoftables

\end{document}